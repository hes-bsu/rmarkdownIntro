\documentclass[11pt,]{article}
\usepackage[left=1in,top=1in,right=1in,bottom=1in]{geometry}
\newcommand*{\authorfont}{\fontfamily{phv}\selectfont}
\usepackage[]{mathpazo}


  \usepackage[T1]{fontenc}
  \usepackage[utf8]{inputenc}
\usepackage{amsmath}
\usepackage{amssymb}



\usepackage{abstract}
\renewcommand{\abstractname}{}    % clear the title
\renewcommand{\absnamepos}{empty} % originally center

\renewenvironment{abstract}
 {{%
    \setlength{\leftmargin}{0mm}
    \setlength{\rightmargin}{\leftmargin}%
  }%
  \relax}
 {\endlist}

\makeatletter
\def\@maketitle{%
  \newpage
%  \null
%  \vskip 2em%
%  \begin{center}%
  \let \footnote \thanks
    {\fontsize{18}{20}\selectfont\raggedright  \setlength{\parindent}{0pt} \@title \par}%
}
%\fi
\makeatother




\setcounter{secnumdepth}{0}

\usepackage{color}
\usepackage{fancyvrb}
\newcommand{\VerbBar}{|}
\newcommand{\VERB}{\Verb[commandchars=\\\{\}]}
\DefineVerbatimEnvironment{Highlighting}{Verbatim}{commandchars=\\\{\}}
% Add ',fontsize=\small' for more characters per line
\usepackage{framed}
\definecolor{shadecolor}{RGB}{248,248,248}
\newenvironment{Shaded}{\begin{snugshade}}{\end{snugshade}}
\newcommand{\AlertTok}[1]{\textcolor[rgb]{0.94,0.16,0.16}{#1}}
\newcommand{\AnnotationTok}[1]{\textcolor[rgb]{0.56,0.35,0.01}{\textbf{\textit{#1}}}}
\newcommand{\AttributeTok}[1]{\textcolor[rgb]{0.77,0.63,0.00}{#1}}
\newcommand{\BaseNTok}[1]{\textcolor[rgb]{0.00,0.00,0.81}{#1}}
\newcommand{\BuiltInTok}[1]{#1}
\newcommand{\CharTok}[1]{\textcolor[rgb]{0.31,0.60,0.02}{#1}}
\newcommand{\CommentTok}[1]{\textcolor[rgb]{0.56,0.35,0.01}{\textit{#1}}}
\newcommand{\CommentVarTok}[1]{\textcolor[rgb]{0.56,0.35,0.01}{\textbf{\textit{#1}}}}
\newcommand{\ConstantTok}[1]{\textcolor[rgb]{0.00,0.00,0.00}{#1}}
\newcommand{\ControlFlowTok}[1]{\textcolor[rgb]{0.13,0.29,0.53}{\textbf{#1}}}
\newcommand{\DataTypeTok}[1]{\textcolor[rgb]{0.13,0.29,0.53}{#1}}
\newcommand{\DecValTok}[1]{\textcolor[rgb]{0.00,0.00,0.81}{#1}}
\newcommand{\DocumentationTok}[1]{\textcolor[rgb]{0.56,0.35,0.01}{\textbf{\textit{#1}}}}
\newcommand{\ErrorTok}[1]{\textcolor[rgb]{0.64,0.00,0.00}{\textbf{#1}}}
\newcommand{\ExtensionTok}[1]{#1}
\newcommand{\FloatTok}[1]{\textcolor[rgb]{0.00,0.00,0.81}{#1}}
\newcommand{\FunctionTok}[1]{\textcolor[rgb]{0.00,0.00,0.00}{#1}}
\newcommand{\ImportTok}[1]{#1}
\newcommand{\InformationTok}[1]{\textcolor[rgb]{0.56,0.35,0.01}{\textbf{\textit{#1}}}}
\newcommand{\KeywordTok}[1]{\textcolor[rgb]{0.13,0.29,0.53}{\textbf{#1}}}
\newcommand{\NormalTok}[1]{#1}
\newcommand{\OperatorTok}[1]{\textcolor[rgb]{0.81,0.36,0.00}{\textbf{#1}}}
\newcommand{\OtherTok}[1]{\textcolor[rgb]{0.56,0.35,0.01}{#1}}
\newcommand{\PreprocessorTok}[1]{\textcolor[rgb]{0.56,0.35,0.01}{\textit{#1}}}
\newcommand{\RegionMarkerTok}[1]{#1}
\newcommand{\SpecialCharTok}[1]{\textcolor[rgb]{0.00,0.00,0.00}{#1}}
\newcommand{\SpecialStringTok}[1]{\textcolor[rgb]{0.31,0.60,0.02}{#1}}
\newcommand{\StringTok}[1]{\textcolor[rgb]{0.31,0.60,0.02}{#1}}
\newcommand{\VariableTok}[1]{\textcolor[rgb]{0.00,0.00,0.00}{#1}}
\newcommand{\VerbatimStringTok}[1]{\textcolor[rgb]{0.31,0.60,0.02}{#1}}
\newcommand{\WarningTok}[1]{\textcolor[rgb]{0.56,0.35,0.01}{\textbf{\textit{#1}}}}
\usepackage{longtable,booktabs}



\title{Models incorporating incomplete reporting improve inferences about
private land conservation \thanks{Template files are available at Steven V. Miller's webpage
(\url{http://svmiller.com/blog/2016/02/svm-r-markdown-manuscript/}).}  }



\author{\Large Matthew A. Williamson\vspace{0.05in} \newline\normalsize\emph{Boise State University}   \and \Large Brett G. Dickson\vspace{0.05in} \newline\normalsize\emph{Conservation Science Partners, Inc.}   \and \Large Mevin B. Hooten\vspace{0.05in} \newline\normalsize\emph{Colorado State University}   \and \Large Rose A. Graves\vspace{0.05in} \newline\normalsize\emph{Portland State University}   \and \Large Mark N. Lubell\vspace{0.05in} \newline\normalsize\emph{University of California, Davis}   \and \Large Mark W. Schwartz\vspace{0.05in} \newline\normalsize\emph{University of California, Davis}  }


\date{}

\usepackage{titlesec}

\titleformat*{\section}{\normalsize\bfseries}
\titleformat*{\subsection}{\normalsize\itshape}
\titleformat*{\subsubsection}{\normalsize\itshape}
\titleformat*{\paragraph}{\normalsize\itshape}
\titleformat*{\subparagraph}{\normalsize\itshape}


\usepackage{natbib}
\bibliographystyle{plainnat}
\usepackage[strings]{underscore} % protect underscores in most circumstances



\newtheorem{hypothesis}{Hypothesis}
\usepackage{setspace}


% set default figure placement to htbp
\makeatletter
\def\fps@figure{htbp}
\makeatother


% move the hyperref stuff down here, after header-includes, to allow for - \usepackage{hyperref}

\makeatletter
\@ifpackageloaded{hyperref}{}{%
\ifxetex
  \PassOptionsToPackage{hyphens}{url}\usepackage[setpagesize=false, % page size defined by xetex
              unicode=false, % unicode breaks when used with xetex
              xetex]{hyperref}
\else
  \PassOptionsToPackage{hyphens}{url}\usepackage[draft,unicode=true]{hyperref}
\fi
}

\@ifpackageloaded{color}{
    \PassOptionsToPackage{usenames,dvipsnames}{color}
}{%
    \usepackage[usenames,dvipsnames]{color}
}
\makeatother
\hypersetup{breaklinks=true,
            bookmarks=true,
            pdfauthor={Matthew A. Williamson (Boise State University) and Brett G. Dickson (Conservation Science Partners, Inc.) and Mevin B. Hooten (Colorado State University) and Rose A. Graves (Portland State University) and Mark N. Lubell (University of California, Davis) and Mark W. Schwartz (University of California, Davis)},
             pdfkeywords = {pandoc, r markdown, knitr},  
            pdftitle={Models incorporating incomplete reporting improve inferences about
private land conservation},
            colorlinks=true,
            citecolor=blue,
            urlcolor=blue,
            linkcolor=magenta,
            pdfborder={0 0 0}}
\urlstyle{same}  % don't use monospace font for urls

% Add an option for endnotes. -----


% add tightlist ----------
\providecommand{\tightlist}{%
\setlength{\itemsep}{0pt}\setlength{\parskip}{0pt}}

% add some other packages ----------

% \usepackage{multicol}
% This should regulate where figures float
% See: https://tex.stackexchange.com/questions/2275/keeping-tables-figures-close-to-where-they-are-mentioned
\usepackage[section]{placeins}


\begin{document}
	
% \pagenumbering{arabic}% resets `page` counter to 1 
%
% \maketitle

{% \usefont{T1}{pnc}{m}{n}
\setlength{\parindent}{0pt}
\thispagestyle{plain}
{\fontsize{18}{20}\selectfont\raggedright 
\maketitle  % title \par  

}

{
   \vskip 13.5pt\relax \normalsize\fontsize{11}{12} 
\textbf{\authorfont Matthew A. Williamson} \hskip 15pt \emph{\small Boise State University}   \par \textbf{\authorfont Brett G. Dickson} \hskip 15pt \emph{\small Conservation Science Partners, Inc.}   \par \textbf{\authorfont Mevin B. Hooten} \hskip 15pt \emph{\small Colorado State University}   \par \textbf{\authorfont Rose A. Graves} \hskip 15pt \emph{\small Portland State University}   \par \textbf{\authorfont Mark N. Lubell} \hskip 15pt \emph{\small University of California, Davis}   \par \textbf{\authorfont Mark W. Schwartz} \hskip 15pt \emph{\small University of California, Davis}   

}

}








\begin{abstract}

    \hbox{\vrule height .2pt width 39.14pc}

    \vskip 8.5pt % \small 

\noindent Developing practical solutions to conservation challenges requires
prioritization approaches that integrate information describing where
conservation should occur with that describing where it does occur.
Empirical evaluation of the arrangement of social, institutional, and
environmental factors that have previously produced conservation actions
is a vital step in moving towards a more complete characterization of
conservation opportunity. Many datasets describing conservation actions
are incomplete, making analyses of predictors of those actions
challenging and potentially prone to bias resulting in
mis-identification of the factors that promote conservation and
hindering the ability to identify locations where future conservation
action may be likely. We adapt the occupancy model framework frequently
deployed in wildlife population studies to the case of partially
reported conservation actions and compare several different formulations
of occupancy models to a naive logistic regression. Through a simulation
study and an empirical evaluation of conservation easements in Idaho and
Montana (United States), we find that occupancy models that explicitly
account for the reporting process produce substantially less-biased
estimates of regression coefficients than logistic regression and are
robust to incomplete separation of the reporting and suitability
process. Results from our case study suggest that occupancy-based models
produced regression coefficient estimates that were more accurate, but
less precise. Occupancy models also resulted in qualitatively different
inferences regarding the effects of predictors we evaluated than those
produced by the naive logistic regression.


\vskip 8.5pt \noindent \emph{Keywords}: pandoc, r markdown, knitr \par

    \hbox{\vrule height .2pt width 39.14pc}



\end{abstract}


\vskip -8.5pt


 % removetitleabstract

\noindent  

\hypertarget{introduction}{%
\section{Introduction}\label{introduction}}

Conserving biodiversity in the face of global change change increasingly
requires conservation scientists and practitioners to prioritize
locations for action (e.g., protection, restoration, or reintroduction
of species). Developing practical solutions to conservation challenges
requires understanding the role of the socio-political system in
constraining or enabling the reduction of threats to species or
ecosystems, the protection of priority areas, or the development of new
conservation tools. Although considerable progress has been made in
identifying where conservation should occur, evaluations of the
socio-ecological conditions under which conservation does occur remain
relatively rare \citep{williamson2018, ban2013}.

Spatially explicit, empirical analyses of the factors that contribute to
the emergence of various institutional (e.g., \citep{lubell2002ws}) and
individual conservation actions (e.g., \citep{metcalf2019, nielsen2017})
provide a critical starting point for evaluating the conditions that
enable conservation to occur. These analyses often use correlative
statistical models (e.g., logistic regression, maximum entropy) to
identify key predictor variables and guide interpretation of regression
coefficients (\textit{sensu} \citealp{lubell2002ws, kroetz2014}). These
analyses take advantage of the growing availability of spatially
extensive, high-resolution data describing social, institutional, and
ecological attributes across a variety of geographies.

\hypertarget{methods}{%
\section{Methods}\label{methods}}

Sometimes you can add some math: \[
\begin{aligned}
y_{ij} &\sim \begin{cases}\operatorname{Bern}(p_{ij}) & z_{i} = 1 \text{ and } v_{ij} = 1\\
0,& z_{i} = 0 \text{ or } v_{ij} = 0
\end{cases},\\
v_{ij} &\sim \begin{cases} \operatorname{Bern}(\alpha_{ij}) & z_{i}=1\\
0, &z_{i} = 0
\end{cases},\\
z_i &\sim \mathrm{Bern}(\psi_i),
\end{aligned}
\] Or you can add a code chunk:

\begin{Shaded}
\begin{Highlighting}[]
\NormalTok{knitr}\OperatorTok{::}\KeywordTok{kable}\NormalTok{(}\KeywordTok{summary}\NormalTok{(iris))}
\end{Highlighting}
\end{Shaded}

\begin{longtable}[]{@{}lccccc@{}}
\toprule
& Sepal.Length & Sepal.Width & Petal.Length & Petal.Width &
Species\tabularnewline
\midrule
\endhead
& Min. :4.300 & Min. :2.000 & Min. :1.000 & Min. :0.100 & setosa
:50\tabularnewline
& 1st Qu.:5.100 & 1st Qu.:2.800 & 1st Qu.:1.600 & 1st Qu.:0.300 &
versicolor:50\tabularnewline
& Median :5.800 & Median :3.000 & Median :4.350 & Median :1.300 &
virginica :50\tabularnewline
& Mean :5.843 & Mean :3.057 & Mean :3.758 & Mean :1.199 &
NA\tabularnewline
& 3rd Qu.:6.400 & 3rd Qu.:3.300 & 3rd Qu.:5.100 & 3rd Qu.:1.800 &
NA\tabularnewline
& Max. :7.900 & Max. :4.400 & Max. :6.900 & Max. :2.500 &
NA\tabularnewline
\bottomrule
\end{longtable}

\newpage
\singlespacing 
\bibliography{/Users/mattwilliamson/Hesweb/rmarkdownIntro/templatedocs/refs.bib}
\end{document}